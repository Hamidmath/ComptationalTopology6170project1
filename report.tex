\documentclass[11pt]{article}
\usepackage[utf8]{inputenc}
\usepackage{geometry}
\geometry{a4paper, margin=1in}
\usepackage{graphicx}
\usepackage{hyperref}
\usepackage{float}
\usepackage{enumitem}

\title{\textbf{Topological Shape Analysis - Comprehensive Project Report}}
\author{}
\date{}

\begin{document}

\maketitle

\section{Project Pipeline \& File Dependencies}
The following details the execution order of the scripts, showing precisely which files act as inputs and which files are generated by each code component.

\subsection*{Step 1: Data Preparation}
\textbf{\texttt{make\_manifest.py}}
\begin{itemize}[label={}]
    \item \textbf{[READS]} Raw image files (\texttt{.gif}) from the \texttt{SelectedSubset/} directory.
    \item \textbf{[WRITES]} A manifest file list to \texttt{data/selected\_files.txt}.
    \item \textbf{[DESC]} Scans the image dataset and creates a consistent list of selected files mapped to their object classes.
\end{itemize}

\subsection*{Step 2: Boundary Extraction}
\textbf{\texttt{extract\_all\_boundaries.py}}
\begin{itemize}[label={}]
    \item \textbf{[READS]} The manifest from \texttt{data/selected\_files.txt} and images from \texttt{SelectedSubset/}.
    \item \textbf{[WRITES]} \texttt{(x, y)} coordinate boundary arrays to the \texttt{txtfiles/} directory (\texttt{*\_points.txt}).
    \item \textbf{[DESC]} Binarizes each shape image, extracts the outermost contour boundary, and saves the 2D point cloud.
\end{itemize}

\textbf{\texttt{draw\_boundary.py}}
\begin{itemize}[label={}]
    \item \textbf{[READS]} A specific boundary file from \texttt{txtfiles/}.
    \item \textbf{[DESC]} A simple utility to visualize a 2D point cloud boundary of a single shape to verify successful extraction.
\end{itemize}

\subsection*{Step 3: Persistent Homology Computation}
\textbf{\texttt{compute\_all\_barcodes.py}}
\begin{itemize}[label={}]
    \item \textbf{[READS]} Boundary coordinate files from \texttt{txtfiles/} (\texttt{*\_points.txt}).
    \item \textbf{[WRITES]} Topological barcodes (H0 and H1 components) to the \texttt{barcodes/} directory (\texttt{*\_H0.txt} and \texttt{*\_H1.txt}).
    \item \textbf{[DESC]} Runs the \texttt{ripser} library on the point clouds to compute their persistent homology up to dimension 1.
\end{itemize}

\textbf{\texttt{draw\_ripser\_barcodes.py}}
\begin{itemize}[label={}]
    \item \textbf{[READS]} A set of barcode files for a specific shape from \texttt{barcodes/}.
    \item \textbf{[DESC]} Visualization utility to plot the H0 and H1 persistence diagrams for a single shape.
\end{itemize}

\subsection*{Step 4: Distance Computation}
\textbf{\texttt{compute\_H0\_bottleneck\_parallel.py}} / \textbf{\texttt{compute\_H1\_bottleneck\_parallel.py}}
\begin{itemize}[label={}]
    \item \textbf{[READS]} H0 or H1 barcodes respectively from the \texttt{barcodes/} directory.
    \item \textbf{[WRITES]} Bottleneck distance matrices and file orders to the \texttt{data/} directory (\texttt{dist\_H0\_bottleneck.npy}, \texttt{H0\_bottleneck\_file\_order.npy}, etc.).
    \item \textbf{[DESC]} Computes pairwise bottleneck distances between topological barcodes in parallel.
\end{itemize}

\textbf{\texttt{compute\_H0\_wasserstein\_parallel.py}} / \textbf{\texttt{compute\_H1\_wasserstein\_parallel.py}}
\begin{itemize}[label={}]
    \item \textbf{[READS]} H0 or H1 barcodes respectively from the \texttt{barcodes/} directory.
    \item \textbf{[WRITES]} Wasserstein distance matrices and file orders to the \texttt{data/} directory.
    \item \textbf{[DESC]} Computes pairwise Wasserstein distances between topological barcodes in parallel.
\end{itemize}

\subsection*{Step 5: Embedding \& Visualization}
\textbf{\texttt{raw\_image\_baseline.py}}
\begin{itemize}[label={}]
    \item \textbf{[READS]} Raw image files from \texttt{SelectedSubset/}.
    \item \textbf{[WRITES]} Euclidean distance baseline matrix (\texttt{data/dist\_raw\_euclidean.npy}) and baseline MDS/t-SNE plots (\texttt{data/mds\_raw\_images.png}, \texttt{data/tsne\_raw\_images.png}).
    \item \textbf{[DESC]} Evaluates a non-topological baseline by computing Euclidean distances between flattened raw pixels and visualizing the embeddings.
\end{itemize}

\textbf{\texttt{mds\_*.py}} \textit{(H0/H1, Bottleneck/Wasserstein)}
\begin{itemize}[label={}]
    \item \textbf{[READS]} Pre-computed distance matrices and file orders from the \texttt{data/} directory.
    \item \textbf{[WRITES]} 2D MDS scatter plot images to the \texttt{data/} directory (e.g., \texttt{mds\_H0\_bottleneck.png}).
    \item \textbf{[DESC]} Applies Multi-Dimensional Scaling (MDS) to layout the shapes based on their topological distances.
\end{itemize}

\textbf{\texttt{tsne\_from\_distance.py}} / \textbf{\texttt{tsne\_*.py}}
\begin{itemize}[label={}]
    \item \textbf{[READS]} A specific topological distance matrix from the \texttt{data/} directory.
    \item \textbf{[WRITES]} 2D t-SNE scatter plot images to the \texttt{data/} directory (e.g., \texttt{tsne\_H0\_bottleneck.png}).
    \item \textbf{[DESC]} Applies t-SNE for an alternative clustering visualization of the distance matrix.
\end{itemize}

\subsection*{Step 6: Baselines \& Machine Learning}
\textbf{\texttt{svm\_persistence\_images.py}}
\begin{itemize}[label={}]
    \item \textbf{[READS]} H1 barcodes from the \texttt{barcodes/} directory.
    \item \textbf{[DESC]} Converts persistence diagrams into grid-based Persistence Images and trains an SVM classifier to predict shape categories.
\end{itemize}

\textbf{\texttt{kernel\_svm\_persistence\_scale\_space.py}}
\begin{itemize}[label={}]
    \item \textbf{[READS]} H1 barcodes from the \texttt{barcodes/} directory.
    \item \textbf{[WRITES]} Computed Persistence Scale Space kernels, class labels, and file lists to the \texttt{results\_kernel/} directory.
    \item \textbf{[DESC]} Computes topological kernel matrices directly from barcodes and trains an SVM. Saves the kernels for further analysis.
\end{itemize}

\textbf{\texttt{kernel\_rep.py}}
\begin{itemize}[label={}]
    \item \textbf{[READS]} Kernel matrices (\texttt{K\_*.npy}) and label/file data from the \texttt{results\_kernel/} directory.
    \item \textbf{[DESC]} An analysis script that verifies matrix structures, min/max values, and class distributions for the generated kernels.
\end{itemize}

\section{Embeddings \& Visualizations}
Below are the resulting MDS and t-SNE projections from the generated distance matrices. These confirm the structural similarities captured by raw pixels vs. topological features.

\subsection*{2.1 Baseline: Raw Images (Euclidean Distance on Pixels)}
\begin{figure}[H]
    \centering
    \begin{minipage}{0.45\textwidth}
        \centering
        \includegraphics[width=\linewidth]{"data/mds_raw_images.png"}
        \caption{MDS Projection - Raw Images}
    \end{minipage}\hfill
    \begin{minipage}{0.45\textwidth}
        \centering
        \includegraphics[width=\linewidth]{"data/tsne_raw_images.png"}
        \caption{t-SNE Projection - Raw Images}
    \end{minipage}
\end{figure}

\subsection*{2.2 Topology: H0 Features (Connected Components)}
\subsubsection*{Bottleneck Distance}
\begin{figure}[H]
    \centering
    \begin{minipage}{0.45\textwidth}
        \centering
        \includegraphics[width=\linewidth]{"data/mds_H0_bottleneck.png"}
        \caption{MDS Projection - H0 Bottleneck}
    \end{minipage}\hfill
    \begin{minipage}{0.45\textwidth}
        \centering
        \includegraphics[width=\linewidth]{"data/tsne_H0_bottleneck.png"}
        \caption{t-SNE Projection - H0 Bottleneck}
    \end{minipage}
\end{figure}

\subsubsection*{Wasserstein Distance}
\begin{figure}[H]
    \centering
    \begin{minipage}{0.45\textwidth}
        \centering
        \includegraphics[width=\linewidth]{"data/mds_H0_wasserstein.png"}
        \caption{MDS Projection - H0 Wasserstein}
    \end{minipage}\hfill
    \begin{minipage}{0.45\textwidth}
        \centering
        \includegraphics[width=\linewidth]{"data/tsne_H0_wasserstein.png"}
        \caption{t-SNE Projection - H0 Wasserstein}
    \end{minipage}
\end{figure}

\subsection*{2.3 Topology: H1 Features (1-Dimensional Holes)}
\subsubsection*{Bottleneck Distance}
\begin{figure}[H]
    \centering
    \begin{minipage}{0.45\textwidth}
        \centering
        \includegraphics[width=\linewidth]{"data/mds_H1_bottleneck.png"}
        \caption{MDS Projection - H1 Bottleneck}
    \end{minipage}\hfill
    \begin{minipage}{0.45\textwidth}
        \centering
        \includegraphics[width=\linewidth]{"data/tsne_H1_bottleneck.png"}
        \caption{t-SNE Projection - H1 Bottleneck}
    \end{minipage}
\end{figure}

\subsubsection*{Wasserstein Distance}
\begin{figure}[H]
    \centering
    \begin{minipage}{0.45\textwidth}
        \centering
        \includegraphics[width=\linewidth]{"data/mds_H1_wasserstein.png"}
        \caption{MDS Projection - H1 Wasserstein}
    \end{minipage}\hfill
    \begin{minipage}{0.45\textwidth}
        \centering
        \includegraphics[width=\linewidth]{"data/tsne_H1_wasserstein.png"}
        \caption{t-SNE Projection - H1 Wasserstein}
    \end{minipage}
\end{figure}

\subsubsection*{Comparison Analysis}
\textbf{Raw vs. Topology:} The baseline embeddings (Raw Images using Euclidean distance on pixels) show heavily scattered classes because raw pixels are highly sensitive to translation, rotation, and slight deformations. The topological embeddings, particularly \textbf{H1 Wasserstein}, demonstrate much tighter clustering for specific classes. Topological features possess deformation and rotation invariance, making them inherently better suited for distinguishing underlying shape morphologies than rigid pixel grids.

\section{Classification Results (Support Vector Machines)}
Two different approaches were used to classify the shapes (8 classes test subset) using topological features.

\subsection*{1. Persistence Images (Grid Feature Vectors)}
\textbf{Accuracy: 25.0\%} \\
Converting the diagrams into fixed-size grid matrices allows for standard linear SVM classification, but resulted in poor accuracy due to the loss of exact spatial geometry and sparse information distribution across the high-dimensional (2288) feature space.

\subsection*{2. Persistence Scale Space Kernel (Direct Diagram Comparison)}
\textbf{Accuracy: 62.5\%} \\
Using a computed topological Gram matrix directly as a Custom SVM kernel dramatically improves performance. This method bypasses grid approximation and computes exact mathematical similarities between the born/die cycles of the shapes across scale spaces.

\subsubsection*{Interpretation of Accuracies}
The jump from 25\% (Images) to 62.5\% (Kernel) highlights a critical truth in Topological Data Analysis (TDA). While vectorization (Persistence Images) is computationally convenient, it fundamentally degrades the structural purity of the barcodes. The \textbf{Scale Space Kernel} preserves the exact distances between topological features, allowing the SVM to find much richer decision boundaries. The classes \texttt{device8} (100\% f1-score) and \texttt{device0} (80\%) were perfectly or near-perfectly recognized by the kernel, indicating their topological signatures (number and lifespan of loops) are highly unique compared to the rest of the dataset.

\end{document}
